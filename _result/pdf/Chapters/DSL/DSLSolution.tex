% -*- mode: latex; -*- mustache tags:  
\documentclass[10pt,twoside,english]{_support/latex/sbabook/sbabook}
\let\wholebook=\relax

\usepackage{import}
\subimport{_support/latex/}{common.tex}

%=================================================================
% Debug packages for page layout and overfull lines
% Remove the showtrims document option before printing
\ifshowtrims
  \usepackage{showframe}
  \usepackage[color=magenta,width=5mm]{_support/latex/overcolored}
\fi


% =================================================================
\title{Learning Object-Oriented Programming, Design and TDD with Pharo}
\author{Stéphane Ducasse}
\series{The Pharo TextBook Collection}

\hypersetup{
  pdftitle = {Learning Object-Oriented Programming, Design and TDD with Pharo},
  pdfauthor = {Stéphane Ducasse},
  pdfkeywords = {Introduction, programming, design, testing, Pharo, Smalltalk}
}


% =================================================================
\begin{document}

% Title page and colophon on verso
\maketitle
\pagestyle{titlingpage}
\thispagestyle{titlingpage} % \pagestyle does not work on the first one…

\cleartoverso
{\small

  Copyright 2017 by Stéphane Ducasse.

  The contents of this book are protected under the Creative Commons
  Attribution-ShareAlike 3.0 Unported license.

  You are \textbf{free}:
  \begin{itemize}
  \item to \textbf{Share}: to copy, distribute and transmit the work,
  \item to \textbf{Remix}: to adapt the work,
  \end{itemize}

  Under the following conditions:
  \begin{description}
  \item[Attribution.] You must attribute the work in the manner specified by the
    author or licensor (but not in any way that suggests that they endorse you
    or your use of the work).
  \item[Share Alike.] If you alter, transform, or build upon this work, you may
    distribute the resulting work only under the same, similar or a compatible
    license.
  \end{description}

  For any reuse or distribution, you must make clear to others the
  license terms of this work. The best way to do this is with a link to
  this web page: \\
  \url{http://creativecommons.org/licenses/by-sa/3.0/}

  Any of the above conditions can be waived if you get permission from
  the copyright holder. Nothing in this license impairs or restricts the
  author's moral rights.

  \begin{center}
    \includegraphics[width=0.2\textwidth]{_support/latex/sbabook/CreativeCommons-BY-SA.pdf}
  \end{center}

  Your fair dealing and other rights are in no way affected by the
  above. This is a human-readable summary of the Legal Code (the full
  license): \\
  \url{http://creativecommons.org/licenses/by-sa/3.0/legalcode}

  \vfill

  % Publication info would go here (publisher, ISBN, cover design…)
  Layout and typography based on the \textcode{sbabook} \LaTeX{} class by Damien
  Pollet.
}


\frontmatter
\pagestyle{plain}

\tableofcontents*
\clearpage\listoffigures

\mainmatter

\chapter{Die DSL }\label{cha:dslsolution}
Here are the possible solutions of the implementation we asked for the DSL Chapter \ref{cha:dsl}. 
\subsection{Define class Die}
\begin{displaycode}{plain}
Object subclass: #Die
	instanceVariableNames: 'faces'
	classVariableNames: ''
	package: 'Dice'
\end{displaycode}

\begin{displaycode}{plain}
Die >> initialize
	super initialize.
	faces := 6
\end{displaycode}
\subsection{Rolling a die}
\begin{displaycode}{plain}
Die >> roll
    ^ faces atRandom
\end{displaycode}
\subsection{Define class DieHandle}
\begin{displaycode}{plain}
Object subclass: #DieHandle
	instanceVariableNames: 'dice'
	classVariableNames: ''
	package: 'Dice'
\end{displaycode}

\begin{displaycode}{plain}
DieHandle >> initialize
	super initialize.
	dice := OrderedCollection new.
\end{displaycode}
\subsection{Die addition}
\begin{displaycode}{plain}
DieHandle >> addDie: aDie 
	dice add: aDie
\end{displaycode}
\section{Rolling a dice handle}
\begin{displaycode}{plain}
DieHandleTest >> testRoll
	| handle |
	handle := DieHandle new
		addDie: (Die withFaces: 6);
		addDie: (Die withFaces: 10);
		yourself.
	1000 timesRepeat: [ handle roll between: 2 and: 16 ]
\end{displaycode}

\begin{displaycode}{plain}
DieHandle >> roll
	
	| res |
	res := 0.
	dice do: [ :each | res := res + each roll ].
	^ res
\end{displaycode}
\section{Role playing syntax}
\begin{displaycode}{plain}
Integer >> D20
	| handle |
	handle := DieHandle new.
	self timesRepeat: [ handle addDie: (Die withFaces: 20)].
	^ handle
\end{displaycode}

\begin{displaycode}{plain}
Integer >> D: anInteger
	
	| handle |
	handle := DieHandle new.
	self timesRepeat: [ handle addDie: (Die withFaces: anInteger)].
	^ handle
\end{displaycode}
\section{Adding DieHandles}
\begin{displaycode}{plain}
DieHandle >> + aDieHandle
	"Returns a new handle that represents the addition of the receiver and the argument."
	| handle |
	handle := self class new.
	self dice do: [ :each | handle addDie: each ].
	aDieHandle dice do: [ :each | handle addDie: each ].
	^ handle
\end{displaycode}

This definition only works if the method \textcode{dice} defined below has been defined

\begin{displaycode}{plain}
DieHandle >> dice
	^ dice 
\end{displaycode}

Indeed the first expression \textcode{self dice do: } could be rewritten as \textcode{dice do: } because dice is an instance variable of the class \textcode{DieHandle}. Now the expression \textcode{aDieHandle dice do: } cannot. Why? Because in Pharo you cannot access the state of another object directly. Here \textcode{2 D20}  is one handle and \textcode{3 D10} another one. The first one cannot access the dice of the second one directly (while it can accessed its own). Therefore there is a need to define a message that provide access to the dice. 


% lulu requires an empty page at the end. That's why I'm using
% \backmatter here.
\backmatter

% Index would go here
\bibliographystyle{abbrv}
\bibliography{others.bib}
\end{document}
