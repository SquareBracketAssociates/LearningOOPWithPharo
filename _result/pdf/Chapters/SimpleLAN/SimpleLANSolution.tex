% -*- mode: latex; -*- mustache tags:  
\documentclass[10pt,twoside,english]{_support/latex/sbabook/sbabook}
\let\wholebook=\relax

\usepackage{import}
\subimport{_support/latex/}{common.tex}

%=================================================================
% Debug packages for page layout and overfull lines
% Remove the showtrims document option before printing
\ifshowtrims
  \usepackage{showframe}
  \usepackage[color=magenta,width=5mm]{_support/latex/overcolored}
\fi


% =================================================================
\title{Learning Object-Oriented Programming, Design and TDD with Pharo}
\author{Stéphane Ducasse}
\series{The Pharo TextBook Collection}

\hypersetup{
  pdftitle = {Learning Object-Oriented Programming, Design and TDD with Pharo},
  pdfauthor = {Stéphane Ducasse},
  pdfkeywords = {Introduction, programming, design, testing, Pharo, Smalltalk}
}


% =================================================================
\begin{document}

% Title page and colophon on verso
\maketitle
\pagestyle{titlingpage}
\thispagestyle{titlingpage} % \pagestyle does not work on the first one…

\cleartoverso
{\small

  Copyright 2017 by Stéphane Ducasse.

  The contents of this book are protected under the Creative Commons
  Attribution-ShareAlike 3.0 Unported license.

  You are \textbf{free}:
  \begin{itemize}
  \item to \textbf{Share}: to copy, distribute and transmit the work,
  \item to \textbf{Remix}: to adapt the work,
  \end{itemize}

  Under the following conditions:
  \begin{description}
  \item[Attribution.] You must attribute the work in the manner specified by the
    author or licensor (but not in any way that suggests that they endorse you
    or your use of the work).
  \item[Share Alike.] If you alter, transform, or build upon this work, you may
    distribute the resulting work only under the same, similar or a compatible
    license.
  \end{description}

  For any reuse or distribution, you must make clear to others the
  license terms of this work. The best way to do this is with a link to
  this web page: \\
  \url{http://creativecommons.org/licenses/by-sa/3.0/}

  Any of the above conditions can be waived if you get permission from
  the copyright holder. Nothing in this license impairs or restricts the
  author's moral rights.

  \begin{center}
    \includegraphics[width=0.2\textwidth]{_support/latex/sbabook/CreativeCommons-BY-SA.pdf}
  \end{center}

  Your fair dealing and other rights are in no way affected by the
  above. This is a human-readable summary of the Legal Code (the full
  license): \\
  \url{http://creativecommons.org/licenses/by-sa/3.0/legalcode}

  \vfill

  % Publication info would go here (publisher, ISBN, cover design…)
  Layout and typography based on the \textcode{sbabook} \LaTeX{} class by Damien
  Pollet.
}


\frontmatter
\pagestyle{plain}

\tableofcontents*
\clearpage\listoffigures

\mainmatter

\chapter{Network simulator solutions}\section{Packets}
\begin{displaycode}{smalltalk}
KANetworkPacket class >> from: sourceAddress to: destinationAddress payload: anObject
    ^ self new
        initializeSource: sourceAddress
        destination: destinationAddress
        payload: anObject
\end{displaycode}

\begin{displaycode}{smalltalk}
KANetworkPacket >> initializeSource: source destination: destination payload: anObject
    sourceAddress := source.
    destinationAddress := destination.
    payload := anObject
\end{displaycode}

\begin{displaycode}{smalltalk}
KANetworkPacket >> sourceAddress
    ^ sourceAddress
\end{displaycode}

\begin{displaycode}{smalltalk}
KANetworkPacket >> destinationAddress
    ^ destinationAddress
\end{displaycode}

\begin{displaycode}{smalltalk}
KANetworkPacket >> payload
    ^ payload
\end{displaycode}
\section{Nodes}
\begin{displaycode}{smalltalk}
KANetworkNode >> initializeAddress: aNetworkAddress
    address := aNetworkAddress
\end{displaycode}

\begin{displaycode}{smalltalk}
KANetworkNode >> address
    ^ address
\end{displaycode}
\section{Links}
\begin{displaycode}{smalltalk}
KANetworkLink >> initializeFrom: sourceNode to: destinationNode
    source := sourceNode.
    destination := destinationNode.
\end{displaycode}

\begin{displaycode}{smalltalk}
KANetworkLink >> source
    ^ source
\end{displaycode}

\begin{displaycode}{smalltalk}
KANetworkLink >> destination
    ^ destination
\end{displaycode}

\begin{displaycode}{smalltalk}
Object subclass: #KANetworkNode
    instanceVariableNames: 'address outgoingLinks'
    classVariableNames: ''
    category: 'NetworkSimulator-Core'
\end{displaycode}

\begin{displaycode}{smalltalk}
KANetworkNode >> hasLinkTo: anotherNode
    ^ outgoingLinks
        anySatisfy: [ :any | any destination == anotherNode ]
\end{displaycode}
\section{Sending a packet}
\begin{displaycode}{smalltalk}
KANetworkLink >> isTransmitting: aPacket
    ^ packetsToTransmit includes: aPacket
\end{displaycode}
\section{Transmitting a packet}
\begin{displaycode}{smalltalk}
KANetworkLink >> transmit: aPacket
    "Transmit aPacket to the destination node of the receiver link."
    (self isTransmitting: aPacket)
        ifTrue: [
            packetsToTransmit remove: aPacket.
            destination receive: aPacket from: self ]
\end{displaycode}

\begin{displaycode}{smalltalk}
Object subclass: #KANetworkNode
    instanceVariableNames: 'address outgoingLinks arrivedPackets'
    classVariableNames: ''
    category: 'NetworkSimulator-Core'
\end{displaycode}

\begin{displaycode}{smalltalk}
KANetworkNode >> initialize
    outgoingLinks := Set new.
    arrivedPackets := OrderedCollection new
\end{displaycode}

\begin{displaycode}{smalltalk}
KANetworkNode >> hasReceived: aPacket
    ^ arrivedPackets includes: aPacket
\end{displaycode}
\section{The loopback link}
\begin{displaycode}{smalltalk}
KANetworkNode >> initialize
    loopback := KANetworkLink from: self to: self.
    outgoingLinks := Set new.
    arrivedPackets := OrderedCollection new
\end{displaycode}

\begin{displaycode}{smalltalk}
KANetworkNode >> linksTowards: anAddress do: aBlock
    "Simple flood algorithm: route via all outgoing links.
    However, just loopback if the receiver node is the routing destination."
    anAddress = address
        ifTrue: [ aBlock value: self loopback ]
        ifFalse: [ outgoingLinks do: aBlock ]
\end{displaycode}
\section{Modeling the network itself}
\begin{displaycode}{smalltalk}
KANetworkTest >> buildNetwork
    alone := KANetworkNode withAddress: #alone.

    net := KANetwork new.

    hub := KANetworkNode withAddress: #hub.
    #(mac pc1 pc2 prn) do: [ :addr |
        | node |
        node := KANetworkNode withAddress: addr.
        net connect: node to: hub ].

    net
        connect: (KANetworkNode withAddress: #ping)
        to: (KANetworkNode withAddress: #pong)
\end{displaycode}

\begin{displaycode}{smalltalk}
KANetwork >> initialize
    nodes := Set new.
    links := Set new
\end{displaycode}

\begin{displaycode}{smalltalk}
KANetwork >> connect: aNode to: anotherNode
    self add: aNode.
    self add: anotherNode.
    links add: (self makeLinkFrom: aNode to: anotherNode) attach.
    links add: (self makeLinkFrom: anotherNode to: aNode) attach
\end{displaycode}
\section{Looking up nodes}
\begin{displaycode}{smalltalk}
KANetwork >> nodeAt: anAddress ifNone: noneBlock
    ^ nodes
        detect: [ :any | any address = anAddress ]
        ifNone: noneBlock
\end{displaycode}
\section{Looking up links}
\begin{displaycode}{smalltalk}
KANetwork >> linkFrom: sourceAddress to: destinationAddress
    ^ links
        detect: [ :anyLink |
            anyLink source address = sourceAddress
                and: [ anyLink destination address = destinationAddress ] ]
        ifNone: [
            NotFound
                signalFor: sourceAddress -> destinationAddress
                in: self ]
\end{displaycode}
\section{Packet delivery with forwarding}
\begin{displaycode}{smalltalk}
KANetworkHub >> forward: aPacket from: arrivalLink
    self
        linksTowards: aPacket destinationAddress
        do: [ :link |
            link destination == arrivalLink source
                ifFalse: [ self send: aPacket via: link ] ]
\end{displaycode}


% lulu requires an empty page at the end. That's why I'm using
% \backmatter here.
\backmatter

% Index would go here
\bibliographystyle{abbrv}
\bibliography{others.bib}
\end{document}
