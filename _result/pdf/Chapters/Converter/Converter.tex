% -*- mode: latex; -*- mustache tags:  
\documentclass[10pt,twoside,english]{_support/latex/sbabook/sbabook}
\let\wholebook=\relax

\usepackage{import}
\subimport{_support/latex/}{common.tex}

%=================================================================
% Debug packages for page layout and overfull lines
% Remove the showtrims document option before printing
\ifshowtrims
  \usepackage{showframe}
  \usepackage[color=magenta,width=5mm]{_support/latex/overcolored}
\fi


% =================================================================
\title{Learning Object-Oriented Programming, Design and TDD with Pharo}
\author{Stéphane Ducasse}
\series{The Pharo TextBook Collection}

\hypersetup{
  pdftitle = {Learning Object-Oriented Programming, Design and TDD with Pharo},
  pdfauthor = {Stéphane Ducasse},
  pdfkeywords = {Introduction, programming, design, testing, Pharo, Smalltalk}
}


% =================================================================
\begin{document}

% Title page and colophon on verso
\maketitle
\pagestyle{titlingpage}
\thispagestyle{titlingpage} % \pagestyle does not work on the first one…

\cleartoverso
{\small

  Copyright 2017 by Stéphane Ducasse.

  The contents of this book are protected under the Creative Commons
  Attribution-ShareAlike 3.0 Unported license.

  You are \textbf{free}:
  \begin{itemize}
  \item to \textbf{Share}: to copy, distribute and transmit the work,
  \item to \textbf{Remix}: to adapt the work,
  \end{itemize}

  Under the following conditions:
  \begin{description}
  \item[Attribution.] You must attribute the work in the manner specified by the
    author or licensor (but not in any way that suggests that they endorse you
    or your use of the work).
  \item[Share Alike.] If you alter, transform, or build upon this work, you may
    distribute the resulting work only under the same, similar or a compatible
    license.
  \end{description}

  For any reuse or distribution, you must make clear to others the
  license terms of this work. The best way to do this is with a link to
  this web page: \\
  \url{http://creativecommons.org/licenses/by-sa/3.0/}

  Any of the above conditions can be waived if you get permission from
  the copyright holder. Nothing in this license impairs or restricts the
  author's moral rights.

  \begin{center}
    \includegraphics[width=0.2\textwidth]{_support/latex/sbabook/CreativeCommons-BY-SA.pdf}
  \end{center}

  Your fair dealing and other rights are in no way affected by the
  above. This is a human-readable summary of the Legal Code (the full
  license): \\
  \url{http://creativecommons.org/licenses/by-sa/3.0/legalcode}

  \vfill

  % Publication info would go here (publisher, ISBN, cover design…)
  Layout and typography based on the \textcode{sbabook} \LaTeX{} class by Damien
  Pollet.
}


\frontmatter
\pagestyle{plain}

\tableofcontents*
\clearpage\listoffigures

\mainmatter

\chapter{Converter}\label{cha_converter}
In this chapter you will implement a little temperature converter between celsius and fahrenheit degrees. It is so simple that it will help us to get started with Pharo and also with test driven development. Near the end of the chapter we will add logging facilities to the converter so that we can log the temperatures of certain locations.  For this you will create a simple class and its tests. 

We will show how to write test to specify the expected results. Writing tests is really important. It is one important tenet of Agile Programming and Test Driven Development (TDD). We will explain later why this is really good to have tests. For now we just implement them. We will also discuss a bit a fundamental aspects of float comparison and we will also present some loops.
\section{First a test}
First we define a test class named \textcode{TemperatureConverterTest} within the package \textcode{MyConverter}. It inherits from the class \textcode{TestCase}. This class is special, any method starting with \textcode{'test'} will be executed automatically, one by one each time on a new instance of the class (to make sure that tests do not interfere with each others). 

\begin{displaycode}{plain}
TestCase subclass: #TemperatureConverterTest
	instanceVariableNames: ''
	classVariableNames: ''
	package: 'MyConverter'
\end{displaycode}

Converting from Fahrenheit to Celsius is done with a simple linear transformation.
The formula to get Fahrenheit from Celsius is F = C * 1.8 + 32. 
Let us write a test covering such transformation. 30 Celsius should be 86 Fahrenheit. 

\begin{displaycode}{plain}
testCelsiusToFahrenheit

	| converter |
	converter := TemperatureConverter new. 
	self assert: ((converter convertCelsius: 30) = 86.0)
\end{displaycode}

The test is structured the following way:

\begin{itemize}
\item Its selector starts with \textcode{test}, here the method is named \textcode{testCelsiusToFahrenheit}.
\item It creates a new instance of \textcode{TemperatureConverter} (it is called the \textit{context} of the test or more technically its fixture).
\item Then we check using the message \textcode{assert:} that the expected behavior is really happening. 
\end{itemize}

The message \textcode{assert:} expects a boolean. Here the expression \textcode{((converter convertCelsius: 30) = 86.0)} returns a boolean. \textcode{true} if the converter returns the value 86.0, \textcode{false} otherwise. 

The testing framework also offers some other similar methods. One is particularly interesting: \textcode{assert:equals:} makes the error reporting more user friendly. The previous method is strictly equivalent to the following one using \textcode{assert:equals:}. 

\begin{displaycode}{plain}
testCelsiusToFahrenheit

	| converter |
	converter := TemperatureConverter new. 
	self assert: (converter convertCelsius: 30) equals: 86.0
\end{displaycode}

The message \textcode{assert:equals:} expects an expression and a result. Here \textcode{(converter convertCelsius: 30)} and \textcode{86.0}. You can use the message you prefer and we suggest to use \textcode{assert:equals:} since it will help you to understand your mistake by saying: \textcode{'You expect 86.0 and I got 30'} instead of simply telling you that the result is \textcode{false}. 
\section{Define a test method (and more)}
While defining the method \textcode{testCelsiusToFahrenheit} using the class browser, the system will tell you that the class \textcode{TemperatureConverter} does not exist (This is true because we did not define it so far). The system will propose to create it. Just let the system do it. 

Once you are done. You should have two classes: \textcode{TemperatureConverterTest} and \textcode{TemperatureConverter}. As well as one method: \textcode{testCelsiusToFahrenheit}. The test does not pass since we did not implement the conversion method (as shown by the red color in the body of \textcode{testCelsiusToFahrenheit}).

Note that you entered the method above and the system compiled it. Now in this book we want to make sure that you know about which method we are talking about hence we will prefix the method definitions with their class. For example the method \textcode{testCelsiusToFahrenheit} in the class \textcode{TemperaturConverterTest} is defined as follows: 

\begin{displaycode}{plain}
TemperaturConverterTest >> testCelsiusToFahrenheit

	| converter |
	converter := TemperatureConverter new. 
	self assert: (converter convertCelsius: 30) equals: 86.0
\end{displaycode}
\section{The class TemperaturConverter}
The class \textcode{TemperaturConverter} is defined as shown below. You could have define it before defining the class  \textcode{TemperaturConverterTest} using the class definition below: 

\begin{displaycode}{plain}
Object subclass: #TemperatureConverter
	instanceVariableNames: ''
	classVariableNames: ''
	package: 'MyConverter'
\end{displaycode}

This definition in essence, says that:

\begin{itemize}
\item We want to define a new class named \textcode{TemperaturConverter}.
\item It has no instance or class variables (\textcode{''} means empty string).
\item It is packaged in package \textcode{MyConverter}.
\end{itemize}

Usually when doing Test Driven Development with Pharo, we focus on tests and lets the system propose us some definitions. Then we can define the method as follows.

\begin{displaycode}{plain}
TemperatureConverter >> convertCelsius: anInteger 
	"Convert anInteger from celsius to fahrenheit"
	
	^ ((anInteger * 1.8) + 32)
\end{displaycode}

The system may tell you that the method is an utility method since it does not use object state. 
It is a bit true because the converter is a \textit{really} simple object. For now do not care. 

Your test should pass. Click on the icon close to the test method to execute it.
\section{Converting from Farhenheit to Celsius }
Now you got the idea. Let us define a test for the conversion from Fahrenheit to Celsius. 

\begin{displaycode}{plain}
TemperatureConverterTest >> testFahrenheitToCelsius

	| converter |
	converter := TemperatureConverter new. 
	self assert: (converter convertFarhenheit: 86) equals: 30.0.
	self assert: (converter convertFarhenheit: 50) equals: 10
\end{displaycode}

Define the method \textcode{convertFarhenheit: anInteger}

\begin{displaycode}{plain}
TemperatureConverter >> convertFarhenheit: anInteger 
	"Convert anInteger from fahrenheit to celsius"
	
	... Your solution ...
\end{displaycode}

Run the tests they should all pass. 
\section{About floats}
The conversions method we wrote returns floats. Floats are special objects in computer science because it is complex to represent infinite information (such as all the numbers between two consecutive integers) with a finite space (numbers are often represented with a fixed number of bits). In particular we should pay attention when comparing 
two floats. Here is a surprising case: we add two floats and the sum is not equal to their sums. 

\begin{displaycode}{plain}
(0.1 + 0.2) = 0.3
> false
\end{displaycode}

This is because the sum is not just equal to \textcode{0.3}. The sum is in fact the number  \textcode{0.30000000000000004}

\begin{displaycode}{plain}
(0.1 + 0.2)
> 0.30000000000000004
\end{displaycode}

To solve this problem in Pharo (it is the same in most programming languages), we do not use equality to compare 
floats but alternate messages such as \textcode{closeTo:} or \textcode{closeTo:precision:} as shown below: 

\begin{displaycode}{plain}
(0.1 + 0.2) closeTo: 0.3
> true
(0.1 + 0.2) closeTo: 0.3 precision: 0.001
> true
\end{displaycode}

To know more, you can have a look at the Fun with Float chapter in Deep Into Pharo (\url{http://books.pharo.org})). The key point is that in computer science you should always avoid to compare the floats naively.

So let us go back to our conversion: 

\begin{displaycode}{plain}
((52 - 32) / 1.8)
> 11.11111111111111
\end{displaycode}

In the following expression we check that the result is close to 11.1 with a precision of 0.1. It means that we
accept as result 11 or 11.1

\begin{displaycode}{plain}
((52 -  32) / 1.8) closeTo: 11.1 precision: 0.1
> true
\end{displaycode}

We can use \textcode{closeTo:precision:} in our tests to make sure that we deal correctly with the float behavior we just described.

\begin{displaycode}{plain}
((52 -  32) / 1.8) closeTo: 11.1 precision: 0.1
> true
\end{displaycode}

We change our tests to reflect this 

\begin{displaycode}{plain}
TemperatureConverterTest >> testFahrenheitToCelsius

	| converter |
	converter := TemperatureConverter new. 
	self assert: ((converter convertFarhenheit: 86) closeTo: 30.0 precision: 0.1).
	self assert: ((converter convertFarhenheit: 50) closeTo: 10 precision: 0.1)
\end{displaycode}
\section{Printing rounded results}
The following expression shows that we may get converted temperature with a too verbose precision. 

\begin{displaycode}{plain}
(TemperatureConverter new convertFarhenheit: 52)
>11.11111111111111
\end{displaycode}

Here just getting 11.1 is enough. There is no need to get the full version. In fact, we can manipulate floats in full precision but there are case like User Interfaces where we would like to get a shorter sort of information. 
Typically as user of the temperature converter, our body does not see the difference between 12.1 or 12.2 degrees. 
Pharo libraries include the message \textcode{printShowingDecimalPlaces: aNumberOfDigit} used to round the \textit{textual} representation of a float.

\begin{displaycode}{plain}
(TemperatureConverter new convertFarhenheit: 52) printShowingDecimalPlaces: 1
>11.1
\end{displaycode}
\section{Building a map of degrees}
Often when you are travelling you would like to have kind of a map of different degrees as follows: 
Here we want to get the converted values between 50 to 70 fahrenheit degrees.

\begin{displaycode}{plain}
(TemperatureConverter new convertFarhenheitFrom: 50 to: 70 by: 2).  
> { 50->10.0. 
	52->11.1. 
	54->12.2. 
	56->13.3. 
	58->14.4. 
	60->15.6. 
	62->16.7. 
	64->17.8. 
	66->18.9. 
	68->20.0. 
	70->21.1}
\end{displaycode}

What we see is that the method \textcode{convertFarhenheitFrom:to:by:} returns an array of pairs.

A pair is created using the message \textcode{-\textgreater{}} and we can access the pair elements using the message \textcode{key} and \textcode{value} as shown below. 

\begin{displaycode}{plain}
| p1 |
p1 := 50 -> 10.0.
p1 key 
>>> 50
p1 value
>>> 10.0
\end{displaycode}

Let us write a test first. We want to generate map containing as key the fahrenheit and as value the converted celsius. Therefore we will get a collection with the map named \textcode{results} and a collection of the expected values
that the value of the elements should have. 

On the two last lines of the test method, using the message \textcode{with:do:} we iterate on both collections in parallel
taking on element of each collection and compare them. 

\begin{displaycode}{plain}
TemperatureConverterTest >> testFToCScale

	| converter results expectedCelsius |
	converter := TemperatureConverter new. 
	results := (converter convertFarhenheitFrom: 50 to: 70 by: 2).
	expectedCelsius := #(10.0 11.1 12.2 13.3 14.4 15.5 16.6 17.7 18.8 20.0 21.1).
	
	results with: expectedCelsius 
		do: [ :res :cel | res value closeTo: cel ] 
\end{displaycode}

Now we are ready to implement the method \textcode{convertFarhenheitFrom: low to: high by: step}.
Using the message \textcode{to:by:}, we create an interval to generate the collection of numbers starting at low and ending up at high using the increment step. Then we use the message \textcode{collect:} which applies a block to a collection 
and returns a collection containing all the values returned by the block application.
Here we just create a pair whose key is the fahrenheit and whose value is its converted celsius value.

\begin{displaycode}{plain}
TemperatureConverter >> convertFarhenheitFrom: low to: high by: step 
	"Returns a collection of pairs (f, c) for all the fahrenheit temperatures from a low to an high temperature"
	
	^ (low to: high by: step)
		collect: [ :f | f -> (self convertFarhenheit: f) ]
\end{displaycode}
\section{Spelling Fahrenheit correctly!}
You may not noticed but we badly spelled fahrenheit since the beginning of this chapter!
Fahrenheit is not spelt farhenheit but Fahrenheit. Now you may start to think that I'm crazy, because you should rename all the methods you wrote and in addition all the users of such methods and after we should check that we did not break anything. And you can think that this is a huge task.

Well first you should rename the methods because nobody wants to keep badly named code. Second, I'm not crazy at all. Programmers rename their code regularly because they often do not get it right the first time, or even the second time... Often you rewrite your code after thinking more about the interface you finally understand that you should propose. In fact good designer think a lot about names because names convey the intent of a computation. Now we have two super powerful tools to help us: Refactorings and Tests. 

We will use the \textbf{Rename method} refactoring proposed by Pharo. A refactoring is a code transformation that preserves code properties. The Rename method refactoring garantees that not only the method but all the places where it is called will also be renamed to send the new message. In addition a refactoring garantees that the behavior of the program is not modified. 
So this is really powerful. 

Select the method \textcode{convertFarhenheit:} in the method list and bring the menu, use the \textbf{Rename method (all)} item, give a new name \textcode{convertFahrenheit:}. The system will prompt you to show you all the corresponding operations. Check them to see what you should have done manually. Imagine the amount of mistakes you could have made and proceed. Do the same for \textcode{convertFahrenheitFrom:to:by:}.

Now the key question is if these changes broke anything. Normally everything should work 
since this is what we expect when using refactorings. But runnning the tests has the final word. So run the tests to check if everything is ok and here is a clear use of tests: they ensure that we can spot fast a regression. 

With this little scenario you should have learned two important things:

\begin{itemize}
\item Tests are written once and executed million times to check for regression. 
\item Refactorings are really powerful operations that save us from tedious manual rewriting.
\end{itemize}
\section{Adding logging behavior}
Imagine now that you want to monitor the different temperatures between the locations where you live and where you work. (This is a real scenario since the building where my office is located got its heating broken over winter and I wanted to measure and keep a trace of the different temperatures in both locations.)

Here is a test representing a typical case.
First, since I want to distinguish my measurements based on the locations, I added the possibility to specify a location. Then I want to be able to record temperatures either in celsius or in fahrenheit. Since the temperature often depends on the moment during the day I want to log the date and time with each measure. 

Then we can request a converter for all the dates (message \textcode{dates}) and temperatures (message \textcode{temperatures}) that it contains.

\begin{displaycode}{plain}
TemperatureConverterTest >> testLocationAndDate

	| office |
	office := TemperatureConverter new location: 'Office'. 
	"perform two measures that are logged"
	office measureCelsius: 19.
	office measureCelsius: 21.
	
	"We got effectively two measures"
	self assert: office measureCount = 2.
	
	"All the measures were done today"
	self assert: office dates equals: {Date today . Date today} asOrderedCollection.
	
	"We logged the correct temperature"
	self assert: office temperatures equals: { 19 . 21 } asOrderedCollection
\end{displaycode}

The first thing that we do is to add two instance variables to our class: \textcode{location} that will hold the name of the location we measure and \textcode{measures} a collection that will hold all the temperatures and dates.

\begin{displaycode}{plain}
Object subclass: #TemperatureConverter
	instanceVariableNames: 'location measures'
	classVariableNames: ''
	package: 'MyConverter'
\end{displaycode}

We initialize such variable with the following values. 

\begin{displaycode}{plain}
TemperatureConverter >> initialize
	location := 'Nice'.
	measures := OrderedCollection new
\end{displaycode}

It means that each instance will be able to have its own location and its own collection of measures. Now we are ready to record a temperature in celsius. 
What we do is that we add pair with the time and the value to our collection of measures. 

\begin{displaycode}{plain}
TemperatureConverter >> measureCelsius: aTemp
	measures add: DateAndTime now -> aTemp
\end{displaycode}

To support tests we also define a method returning the number of current measure our instance holds. 

\begin{displaycode}{plain}
TemperatureConverter >> measureCount
	... Your code ...
\end{displaycode}

We now define three methods returning the sequence of temperatures, the dates and the times. 
Since the time has a microsecond precision it is a bit difficult to test. So we only test the dates. 

\begin{displaycode}{plain}
TemperatureConverter >> temperatures
	^ measures collect: [ :each | each value ]
\end{displaycode}

To produce time without micro second we suggest to print the time using \textcode{print24}.

\begin{displaycode}{plain}
DateAndTime now asTime print24
>>> '22:46:33'
\end{displaycode}

\begin{displaycode}{plain}
TemperatureConverter >> times
	^ measures collect: [ :each | each key asTime ]
\end{displaycode}

\begin{displaycode}{plain}
TemperatureConverter >> dates
	... Your code ...
\end{displaycode}

Now we can get two converters each with its own location and measurement records. 
The following tests verify that this is the case. 

\begin{displaycode}{plain}
TemperatureConverterTest >> testTwoLocations

	| office home |
	office := TemperatureConverter new location: 'office'. 
	office measureFahrenheit: 19.
	office measureFahrenheit: 21.
	self assert: office location equals: 'office'.
	self assert: office measureCount equals: 2. 
	home := TemperatureConverter new location: 'home'. 
	home measureFahrenheit: 22.
	home measureFahrenheit: 22.
	home measureFahrenheit: 22.
	self assert: home location equals: 'home'.
	self assert: home measureCount equals: 3.
\end{displaycode}

We can add now a new method to convert fahrenheit to celcius and we define a new test first.

\begin{displaycode}{plain}
TemperatureConverterTest >> testLocationAndDateWithConversion

	| converter |
	converter := TemperatureConverter new location: 'Lille'. 
	converter measureFahrenheit: 86.
	converter measureFahrenheit: 50.
	self assert: converter measureCount equals: 2.
	self assert: converter dates 
		equals: {Date today . Date today} asOrderedCollection.
	self assert: converter temperatures 
		equals: { converter convertFahrenheit: 86 . 
				converter convertFahrenheit: 50 } asOrderedCollection
\end{displaycode}

What we do is that since celsius is the scientific unity for temperature we convert to celsius before recording our temperature. 

\begin{displaycode}{plain}
TemperatureConverter >> measureFahrenheit: aTemp
	... Your code ...
\end{displaycode}
\section{Discussion }
From a design perspective we see that the logger behavior is a much better object than the converter. The logger keeps some internal data specific to a location while the converter is stateless. Object-oriented programming is much better for capturing object with state. This is why the converter was a kind of silly objects but it was to get you started. 
Now it is rare that the world we want to model and represent is stateless. This is why object-oriented programming is a powerful way to develop complex programs.
\section{Conclusion}
In this chapter we built a simple temperature converter. We showed how define and execute unit tests using a Test Driven approach. The interest in testing and Test Driven Development is not limited to Pharo. Automated testing has become a hallmark of the \textit{Agile software development} movement, and any software developer concerned with improving
software quality would do well to adopt it.

We showed that tests are an important aid to measure our progress and also are an important aid to define clearly what we want to develop.



% lulu requires an empty page at the end. That's why I'm using
% \backmatter here.
\backmatter

% Index would go here
\bibliographystyle{abbrv}
\bibliography{others.bib}
\end{document}
