% -*- mode: latex; -*- mustache tags:  
\documentclass[10pt,twoside,english]{_support/latex/sbabook/sbabook}
\let\wholebook=\relax

\usepackage{import}
\subimport{_support/latex/}{common.tex}

%=================================================================
% Debug packages for page layout and overfull lines
% Remove the showtrims document option before printing
\ifshowtrims
  \usepackage{showframe}
  \usepackage[color=magenta,width=5mm]{_support/latex/overcolored}
\fi


% =================================================================
\title{Learning Object-Oriented Programming, Design and TDD with Pharo}
\author{Stéphane Ducasse}
\series{The Pharo TextBook Collection}

\hypersetup{
  pdftitle = {Learning Object-Oriented Programming, Design and TDD with Pharo},
  pdfauthor = {Stéphane Ducasse},
  pdfkeywords = {Introduction, programming, design, testing, Pharo, Smalltalk}
}


% =================================================================
\begin{document}

% Title page and colophon on verso
\maketitle
\pagestyle{titlingpage}
\thispagestyle{titlingpage} % \pagestyle does not work on the first one…

\cleartoverso
{\small

  Copyright 2017 by Stéphane Ducasse.

  The contents of this book are protected under the Creative Commons
  Attribution-ShareAlike 3.0 Unported license.

  You are \textbf{free}:
  \begin{itemize}
  \item to \textbf{Share}: to copy, distribute and transmit the work,
  \item to \textbf{Remix}: to adapt the work,
  \end{itemize}

  Under the following conditions:
  \begin{description}
  \item[Attribution.] You must attribute the work in the manner specified by the
    author or licensor (but not in any way that suggests that they endorse you
    or your use of the work).
  \item[Share Alike.] If you alter, transform, or build upon this work, you may
    distribute the resulting work only under the same, similar or a compatible
    license.
  \end{description}

  For any reuse or distribution, you must make clear to others the
  license terms of this work. The best way to do this is with a link to
  this web page: \\
  \url{http://creativecommons.org/licenses/by-sa/3.0/}

  Any of the above conditions can be waived if you get permission from
  the copyright holder. Nothing in this license impairs or restricts the
  author's moral rights.

  \begin{center}
    \includegraphics[width=0.2\textwidth]{_support/latex/sbabook/CreativeCommons-BY-SA.pdf}
  \end{center}

  Your fair dealing and other rights are in no way affected by the
  above. This is a human-readable summary of the Legal Code (the full
  license): \\
  \url{http://creativecommons.org/licenses/by-sa/3.0/legalcode}

  \vfill

  % Publication info would go here (publisher, ISBN, cover design…)
  Layout and typography based on the \textcode{sbabook} \LaTeX{} class by Damien
  Pollet.
}


\frontmatter
\pagestyle{plain}

\tableofcontents*
\clearpage\listoffigures

\mainmatter

\chapter{Solution of challenge yourself}\section{Challenge: Message identification}
\begin{displaycode}{plain}
3 + 4

	receiver: 3
	selector: +
	arguments: 4
	result: 7
\end{displaycode}

\begin{displaycode}{plain}
Date today

	receiver: Date
	selector: today 
	arguments: _
	result: The date of today
\end{displaycode}

\begin{displaycode}{plain}
#('' 'World') at: 1 put: 'Hello'

	receiver: #('' 'World')
	selector: at:put:
	arguments: 1 and 'Hello'
	result: #('Hello' 'World')
\end{displaycode}

\begin{displaycode}{plain}
#(1 22 333) at: 2

	receiver: #(1 22 333)
	selector:	at:
	arguments: 2
	result: 22
\end{displaycode}

\begin{displaycode}{plain}
#(2 33 -4 67) collect: [ :each | each abs ]

	receiver: #(2 33 -4 67)
	selector: collect: 
	arguments: [ :each | each abs ]
	result: #(2 33 4 67)
\end{displaycode}

\begin{displaycode}{plain}
25 @ 50

	receiver: 25
	selector: @
	arguments: 50
	result: 25@50 (a point)
\end{displaycode}

\begin{displaycode}{plain}
SmallInteger maxVal


	receiver: the class SmalltalkInteger
	selector: maxVal
	arguments: _ 
	result: returns the largest small integer
\end{displaycode}

\begin{displaycode}{plain}
#(a b c d e f) includesAll: #(f d b)

	receiver: #(a b c d e f)
	selector: includesAll:
	arguments: #(f d b)
	result: true
\end{displaycode}

\begin{displaycode}{plain}
true | false

	receiver: true
	selector: |
	arguments: false
	result: true
\end{displaycode}

\begin{displaycode}{plain}
Point selectors

	receiver: Point 
	selector: selectors
	arguments: _
	result: a long arrays of selectors understood by the class Point 
\end{displaycode}
\section{Challenge: Literal objects}
What kind of object does the following literal expressions refer to? It is the same as asking what is the result of sending the \textcode{class} message to such expressions. 

\begin{displaycode}{plain}
1.3

> Float
\end{displaycode}

\begin{displaycode}{plain}
#node1

> Symbol
\end{displaycode}

\begin{displaycode}{plain}
#(2 33 4)

> Array
\end{displaycode}

\begin{displaycode}{plain}
'Hello, Dave'

> String
\end{displaycode}

\begin{displaycode}{plain}
[ :each | each scale: 1.5 ]

> Block
\end{displaycode}

\begin{displaycode}{plain}
$A 

> Character
\end{displaycode}

\begin{displaycode}{plain}
true

> Boolean
\end{displaycode}

\begin{displaycode}{plain}
1

> SmallInteger
\end{displaycode}
\section{Challenge: Kind of messages}
Examine the following messages and report if the message is unary, binary or keyword-based.

\begin{displaycode}{plain}
1 log

> Unary
\end{displaycode}

\begin{displaycode}{plain}
Browser open

> Unary
\end{displaycode}

\begin{displaycode}{plain}
2 raisedTo: 5

> Keyword-based
\end{displaycode}

\begin{displaycode}{plain}
'hello', 'world'

> Binary
\end{displaycode}

\begin{displaycode}{plain}
10@20

> Binary
\end{displaycode}

\begin{displaycode}{plain}
point1 x

> Unary
\end{displaycode}

\begin{displaycode}{plain}
point1 distanceFrom: point2

> Keyword-based
\end{displaycode}
\section{Challenge: Results}
Examine the following expressions. What is the value returned by the execution of the following expressions?

\begin{displaycode}{plain}
1 + 3 negated

> -2
\end{displaycode}

\begin{displaycode}{plain}
1 + (3 negated)

> -2
\end{displaycode}

\begin{displaycode}{plain}
2 raisedTo: 3 + 2

> 32
\end{displaycode}

\begin{displaycode}{plain}
| anArray |
anArray := #('first' 'second' 'third' 'fourth').
anArray at: 2


> 'second'
\end{displaycode}

\begin{displaycode}{plain}
#(2 3 -10 3) collect: [ :each | each * each]

> #(4 9 100 9)
\end{displaycode}

\begin{displaycode}{plain}
6 + 4 / 2

> 5
\end{displaycode}

\begin{displaycode}{plain}
2 negated raisedTo: 3 + 2

> -32
\end{displaycode}

\begin{displaycode}{plain}
#(a b c d e f) includesAll: #(f d b)

> true
\end{displaycode}
\section{Challenge: unneeded parentheses }
Putting more  parentheses than necessary is  a good way  to get started. Such  practice however leads to  less readable expressions. Rewrite the following expressions using the least number of parentheses. 

\begin{displaycode}{plain}
x between: (pt1 x) and: (pt2 y)

is equivalent to 

x between: pt1 x and: pt2 y
\end{displaycode}

\begin{displaycode}{plain}
((#(a b c d e f) asSet) intersection: (#(f d b) asSet))

is equivalent to 

#(a b c d e f) asSet intersection: #(f d b) asSet
\end{displaycode}

\begin{displaycode}{plain}
(x isZero)
     ifTrue: [....]
(x includes: y)
     ifTrue: [....]
	 
is equivalent to 


x isZero
     ifTrue: [....]
(x includes: y)
     ifTrue: [....]
\end{displaycode}

\begin{displaycode}{plain}
(OrderedCollection new)
    add: 56; 
    add: 33; 
    yourself

is equivalent to

OrderedCollection new
	    add: 56; 
	    add: 33; 
	    yourself
\end{displaycode}

\begin{displaycode}{plain}
((3 + 4) + (2 * 2) + (2 * 3))

is equivalent to

3 + 4 + (2 * 2) + (2 * 3)
\end{displaycode}

\begin{displaycode}{plain}
(Integer primesUpTo: 64) sum

No changes
\end{displaycode}

\begin{displaycode}{plain}
('http://www.pharo.org' asUrl) retrieveContents

is equivalent to 

'http://www.pharo.org' asUrl retrieveContents
\end{displaycode}


% lulu requires an empty page at the end. That's why I'm using
% \backmatter here.
\backmatter

% Index would go here
\bibliographystyle{abbrv}
\bibliography{others.bib}
\end{document}
