% -*- mode: latex; -*- mustache tags:  
\documentclass[10pt,twoside,english]{_support/latex/sbabook/sbabook}
\let\wholebook=\relax

\usepackage{import}
\subimport{_support/latex/}{common.tex}

%=================================================================
% Debug packages for page layout and overfull lines
% Remove the showtrims document option before printing
\ifshowtrims
  \usepackage{showframe}
  \usepackage[color=magenta,width=5mm]{_support/latex/overcolored}
\fi


% =================================================================
\title{Learning Object-Oriented Programming, Design and TDD with Pharo}
\author{Stéphane Ducasse}
\series{The Pharo TextBook Collection}

\hypersetup{
  pdftitle = {Learning Object-Oriented Programming, Design and TDD with Pharo},
  pdfauthor = {Stéphane Ducasse},
  pdfkeywords = {Introduction, programming, design, testing, Pharo, Smalltalk}
}


% =================================================================
\begin{document}

% Title page and colophon on verso
\maketitle
\pagestyle{titlingpage}
\thispagestyle{titlingpage} % \pagestyle does not work on the first one…

\cleartoverso
{\small

  Copyright 2017 by Stéphane Ducasse.

  The contents of this book are protected under the Creative Commons
  Attribution-ShareAlike 3.0 Unported license.

  You are \textbf{free}:
  \begin{itemize}
  \item to \textbf{Share}: to copy, distribute and transmit the work,
  \item to \textbf{Remix}: to adapt the work,
  \end{itemize}

  Under the following conditions:
  \begin{description}
  \item[Attribution.] You must attribute the work in the manner specified by the
    author or licensor (but not in any way that suggests that they endorse you
    or your use of the work).
  \item[Share Alike.] If you alter, transform, or build upon this work, you may
    distribute the resulting work only under the same, similar or a compatible
    license.
  \end{description}

  For any reuse or distribution, you must make clear to others the
  license terms of this work. The best way to do this is with a link to
  this web page: \\
  \url{http://creativecommons.org/licenses/by-sa/3.0/}

  Any of the above conditions can be waived if you get permission from
  the copyright holder. Nothing in this license impairs or restricts the
  author's moral rights.

  \begin{center}
    \includegraphics[width=0.2\textwidth]{_support/latex/sbabook/CreativeCommons-BY-SA.pdf}
  \end{center}

  Your fair dealing and other rights are in no way affected by the
  above. This is a human-readable summary of the Legal Code (the full
  license): \\
  \url{http://creativecommons.org/licenses/by-sa/3.0/legalcode}

  \vfill

  % Publication info would go here (publisher, ISBN, cover design…)
  Layout and typography based on the \textcode{sbabook} \LaTeX{} class by Damien
  Pollet.
}


\frontmatter
\pagestyle{plain}

\tableofcontents*
\clearpage\listoffigures

\mainmatter

\chapter{Electronic wallet solution}\label{cha:walletSol}
Here are the possible solutions of the implementation we asked for the Wallet Chapter \ref{cha:wallet}. 
\section{Using a bag for a wallet}
\begin{displaycode}{plain}
Wallet >> add: anInteger coinsOfValue: aCoin
	"Add to the receiver, anInteger times a coin of value aNumber"

	bagCoins add: aCoin withOccurrences: anInteger 
\end{displaycode}

We can add elements one by one to a bag using the message \textcode{add:} or specifying the number of occurences of the element using the message \textcode{add:withOccurrences:}. 

\begin{displaycode}{plain}
Wallet >> coinsOfValue: aNumber

	^ bagCoins occurrencesOf: aNumber
\end{displaycode}
\section{Testing money}
\begin{displaycode}{plain}
Wallet >> money
	"Return the value of the receiver by summing its constituents"
	| money |
	money := 0.
	bagCoins doWithOccurrences:
			[ :elem : occurrence | 
				money := money + ( elem * occurrence ) ].
	^ money
\end{displaycode}
\section{Checking to pay an amount}
\begin{displaycode}{plain}
Wallet >> canPay: amounOfMoney
	"returns true when we can pay the amount of money"
	^ self money >= amounOfMoney
\end{displaycode}
\section{Biggest coin}
\begin{displaycode}{plain}
Wallet >> biggest
	"Returns the biggest coin with a value below anAmount. For example, {(3 -> 0.5) . (3 -> 0.2) . (5-> 0.1)} biggest -> 0.5"

	^ bagCoins sortedElements last key
\end{displaycode}
\section{Biggest below a value}
\begin{displaycode}{plain}
Wallet >> biggestBelow: anAmount
	"Returns the biggest coin with a value below anAmount. For example, {(3 -> 0.5) . (3 -> 0.2) . (5-> 0.1)} biggestBelow: 0.40 -> 0.2"
	
	bagCoins doWithOccurrences: [ :elem :occurrences |
			anAmount > elem ifTrue: [ ^ elem ] ].
	^ 0
\end{displaycode}
\section{Improving the API}
\begin{displaycode}{plain}
Wallet >> addCoin: aNumber
	"Add to the receiver a coin of value aNumber"
	
	bagCoins add: aNumber withOccurrences: 1 
\end{displaycode}

\begin{displaycode}{plain}
Wallet >> removeCoin: aNumber
	"Remove from the receiver a coin of value aNumber"
	
	bagCoins remove: aNumber ifAbsent: [] 
\end{displaycode}
\section{Coins for paying: First version}
\begin{displaycode}{plain}
Wallet >> coinsFor: aValue
	"Returns a wallet with the largest coins to pay a certain amount and an empty wallet if this is not possible"
	| res |
	res := self class new.
	^ (self canPay: aValue)
		ifFalse: [ res ]
		ifTrue: [ self coinsFor: aValue into2: res ] 
\end{displaycode}

\begin{displaycode}{plain}
Wallet >> coinsFor: aValue into2: accuWallet
	| accu |
	[ accu := accuWallet money.
	accu < aValue ]
		whileTrue: [
			| big |
			big := self biggest.
			[ big > ((aValue - accu) roundUpTo: 0.1) ] 
				whileTrue: [ big := self biggestBelow: big ].
			self removeCoin: big.
			accuWallet addCoin: big ].
	^ accuWallet 
\end{displaycode}


% lulu requires an empty page at the end. That's why I'm using
% \backmatter here.
\backmatter

% Index would go here
\bibliographystyle{abbrv}
\bibliography{others.bib}
\end{document}
