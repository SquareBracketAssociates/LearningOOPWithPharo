% -*- mode: latex; -*- mustache tags:  
\documentclass[10pt,twoside,english]{_support/latex/sbabook/sbabook}
\let\wholebook=\relax

\usepackage{import}
\subimport{_support/latex/}{common.tex}

%=================================================================
% Debug packages for page layout and overfull lines
% Remove the showtrims document option before printing
\ifshowtrims
  \usepackage{showframe}
  \usepackage[color=magenta,width=5mm]{_support/latex/overcolored}
\fi


% =================================================================
\title{Learning Object-Oriented Programming, Design and TDD with Pharo}
\author{Stéphane Ducasse}
\series{The Pharo TextBook Collection}

\hypersetup{
  pdftitle = {Learning Object-Oriented Programming, Design and TDD with Pharo},
  pdfauthor = {Stéphane Ducasse},
  pdfkeywords = {Introduction, programming, design, testing, Pharo, Smalltalk}
}


% =================================================================
\begin{document}

% Title page and colophon on verso
\maketitle
\pagestyle{titlingpage}
\thispagestyle{titlingpage} % \pagestyle does not work on the first one…

\cleartoverso
{\small

  Copyright 2017 by Stéphane Ducasse.

  The contents of this book are protected under the Creative Commons
  Attribution-ShareAlike 3.0 Unported license.

  You are \textbf{free}:
  \begin{itemize}
  \item to \textbf{Share}: to copy, distribute and transmit the work,
  \item to \textbf{Remix}: to adapt the work,
  \end{itemize}

  Under the following conditions:
  \begin{description}
  \item[Attribution.] You must attribute the work in the manner specified by the
    author or licensor (but not in any way that suggests that they endorse you
    or your use of the work).
  \item[Share Alike.] If you alter, transform, or build upon this work, you may
    distribute the resulting work only under the same, similar or a compatible
    license.
  \end{description}

  For any reuse or distribution, you must make clear to others the
  license terms of this work. The best way to do this is with a link to
  this web page: \\
  \url{http://creativecommons.org/licenses/by-sa/3.0/}

  Any of the above conditions can be waived if you get permission from
  the copyright holder. Nothing in this license impairs or restricts the
  author's moral rights.

  \begin{center}
    \includegraphics[width=0.2\textwidth]{_support/latex/sbabook/CreativeCommons-BY-SA.pdf}
  \end{center}

  Your fair dealing and other rights are in no way affected by the
  above. This is a human-readable summary of the Legal Code (the full
  license): \\
  \url{http://creativecommons.org/licenses/by-sa/3.0/legalcode}

  \vfill

  % Publication info would go here (publisher, ISBN, cover design…)
  Layout and typography based on the \textcode{sbabook} \LaTeX{} class by Damien
  Pollet.
}


\frontmatter
\pagestyle{plain}

\tableofcontents*
\clearpage\listoffigures

\mainmatter

\chapter{Summing and converting money}
We will now work on one example proposed by A. Bergel and we would like to thank him for it. 

\begin{displaycode}{plain}
1 EUR = 662 CLP (Chilean pesos) 
\end{displaycode}
\section{Requirements}
	

\begin{displaycode}{plain}
TestCase subclass: #CurrencyTest 
\end{displaycode}

\begin{displaycode}{plain}
CurrencyTest >> testSum 
   | clp1 eur1 clp2 eur2 |
   clp1 := CLP new value: 3500.
   eur1 := EUR new value: 10.
   clp2 := CLP new value: 5000.
   eur2 := EUR new value: 20.

   self assert: clp1 + clp2 equals: (CLP new value: 8500). 
   self assert: clp1 + eur1 equals: (CLP new value: 3500 + 6620).
   
   self assert: eur1 + eur2 equals: (EUR new value: 30).
   self assert: eur1 + clp2 equals: (EUR new value: 5000 / 662 + 10).
\end{displaycode}

In addition in a second step we will add conversion between Euros and USD. 
\section{Given context}
You have a class \textcode{Currency} to which you can sum other currencyCurrency. 

\begin{displaycode}{plain}
Object subclass: #Currency 
	instVarNames: ‘value’ 
\end{displaycode}

\begin{displaycode}{plain}
Currency >> + anotherCurrency 
   self subclassResponsibility 
\end{displaycode}

\begin{displaycode}{plain}
Currency >> printOn: str
   super printOn: str.
   str nextPut: $<.
   str nextPutAll: self value asString.
   str nextPut: $>.
\end{displaycode}
\section{Solution}
\begin{displaycode}{plain}
Currency >> sumWithEUR: money 
   self subclassResponsibility 
\end{displaycode}

\begin{displaycode}{plain}
Currency >> sumWithCLP: money 
   self subclassResponsibility 
\end{displaycode}

\begin{displaycode}{plain}
Currency >> = anotherCurrency 
  ^ self class == anotherCurrency class and: [ self value = anotherCurrency value ]
\end{displaycode}

You have two subclasses: 

\begin{displaycode}{plain}
Currency subclass: #EUR 

Currency subclass: #CLP 
\end{displaycode}

\begin{displaycode}{plain}
EUR >> + anotherCurrency 
   ^ anotherCurrency sumWithEUR: self
\end{displaycode}

\begin{displaycode}{plain}
EUR >> sumWithEUR: money 
   ^ EUR new value: self value + money value
\end{displaycode}

\begin{displaycode}{plain}
EUR >> sumWithCLP: money 
   ^ CLP new value: (self value * 662) + money value
\end{displaycode}

\begin{displaycode}{plain}
CLP >> + anotherCurrency
   ^ anotherCurrency sumWithCLP: self
\end{displaycode}

\begin{displaycode}{plain}
CLP >> sumWithEUR: money
   ^ EUR new value: (self value / 662) + money value
\end{displaycode}

\begin{displaycode}{plain}
CLP >> sumWithCLP: money
  ^ CLP new value: self value + money value
\end{displaycode}


% lulu requires an empty page at the end. That's why I'm using
% \backmatter here.
\backmatter

% Index would go here
\bibliographystyle{abbrv}
\bibliography{others.bib}
\end{document}
